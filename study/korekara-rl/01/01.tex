\documentclass[dvipdfmx,cjk]{beamer}
%\documentclass[dvipdfm,cjk]{beamer}  %% オプションは環境や利用するプログラムに
%\documentclass[dvips,cjk]{beamer}   %% よって変える

%\AtBeginDvi{\special{pdf:tounicode 90ms-RKSJ-UCS2}} %% しおりが文字化けしないように
\AtBeginDvi{\special{pdf:tounicode EUC-UCS2}}

%\setbeamertemplate{navigation symbols}{} %% 右下のアイコンを消す

\usetheme{CambridgeUS}         %% theme の選択
%\usetheme{Boadilla}           %% Beamer のディレクトリの中の
%\usetheme{Madrid}             %% beamerthemeCambridgeUS.sty を指定
%\usetheme{Antibes}            %% 色々と試してみるといいだろう
%\usetheme{Montpellier}        %% サンプルが beamer\doc に色々とある。
%\usetheme{Berkeley}
%\usetheme{Goettingen}
%\usetheme{Singapore}
%\usetheme{Szeged}

%\usecolortheme{rose}          %% colortheme を選ぶと色使いが変わる
%\usecolortheme{albatross}

%\useoutertheme{shadow}                 %% 箱に影をつける
%\usefonttheme{professionalfonts}       %% 数式の文字を通常の LaTeX と同じにする

%\setbeamercovered{transparent}         %% 消えている文字をうっすらと表示する

\setbeamertemplate{theorems}[numbered]  %% 定理に番号をつける
\newtheorem{thm}{Theorem}[section]
\newtheorem{proposition}[thm]{Proposition}
\theoremstyle{example}
\newtheorem{exam}[thm]{Example}
\newtheorem{remark}[thm]{Remark}
\newtheorem{question}[thm]{Question}
\newtheorem{prob}[thm]{Problem}

\begin{document}
\title[Beamer]{強化学習導入} 
\author[ManabuNishiura]{西浦学}            %% ここに書かれた情報は色々なところに使われるので
\institute[The University of Tokyo]{東京大学}   %% なるべく設定した方が良い
\date{July N, 2020}

\begin{frame}                  %% \begin{frame}..\end{frame} で 1 枚のスライド
\titlepage                     %% タイトルページ
\end{frame}

\begin{frame}{}                  %% 目次 (必要なければ省略)
\tableofcontents
\end{frame}

\section{Characters in RL world}             %% セクション名
\begin{frame}
\frametitle{Value function}              %% フレームタイトル
\begin{aligned} V ^ { \pi } ( s ) & = \mathbb { E } ^ { \pi } \left[ G _ { t } | S _ { t } = s \right] \\ & = \sum _ { s ^ { \prime } \in \mathcal { S } } \sum _ { a \in \mathcal { A } ( s ) } P \left( S _ { t + 1 } = s ^ { \prime } , A _ { t } = a | S _ { t } = s \right) r \left( s , a , s ^ { \prime } \right) \\ & = \sum _ { s ^ { \prime } \in \mathcal { S } } \sum _ { a \in \mathcal { A } ( s ) } P \left( S _ { t + 1 } = s ^ { \prime } | S _ { t } = s , A _ { t } = a \right) \pi ( a | s ) r \left( s , a , s ^ { \prime } \right) \end{aligned}
\end{frame}

\begin{frame}
an example of finite horizon return $T=2$
    \begin{equation}
        G_t = R_{t+1} + R_{t+2}
    \end{equation}
therefore the value function can be calculated as following.
\begin{align*}
V^{\pi} ( s ) = & \mathbb { E } ^ { \pi } \left[ G _ { t } | S _ { t } = s \right] = \mathbb { E } ^ { \pi } \left[ R _ { t + 1 } + R _ { t + 2 } | S _ { t } = s \right] \nonumber\\ 
    & = \sum _ { s ^ { \prime \prime } \in \mathcal { S } } \sum _ { a ^ { \prime } \in \mathcal { A } ( s ) } \sum _ { s ^ { \prime } \in \mathcal { S } } \sum _ { a \in \mathcal { A } ( s ) } \nonumber \\
    & P \left( S _ { t + 2 } = s ^ { \prime \prime } , A _ { t + 1 } = a ^ { \prime } , S _ { t + 1 } = s ^ { \prime } , A _ { t } = a | S _ { t } = s \right) \nonumber\\ 
    & \times \left\{ r \left( s , a , s ^ { \prime } \right) + r \left( s ^ { \prime } , a ^ { \prime } , s ^ { \prime \prime } \right) \right\} \nonumber\\ 
    & = \sum _ { s ^ { \prime \prime } \in \mathcal { S } } \sum _ { a ^ { \prime } \in \mathcal { A } ( s ) } \sum _ { s ^ { \prime } \in \mathcal { S } } \sum _ { a \in \mathcal { A } ( s ) } \nonumber\\
    & P \left( S _ { t + 2 } = s ^ { \prime \prime } | S _ { t + 1 } = s ^ { \prime } , A _ { t + 1 } = a ^ { \prime } \right) \pi \left( a ^ { \prime } | s ^ { \prime } \right) \nonumber\\ 
    &\times P \left( S _ { t + 1 } = s ^ { \prime } | S _ { t } = s , A _ { t } = a \right) \pi ( a | s ) \left\{ r \left( s , a , s ^ { \prime } \right) + r \left( s ^ { \prime } , a ^ { \prime } , s ^ { \prime \prime } \right) \right\}
\end{align*}
\end{frame}

\section{policy gradient theorem}           %% 定理型環境が使える
\begin{frame}                  %% \newtheorem で新しい環境も作れる
\begin{thm}
定理型環境が使える。
使い方は普通の \LaTeX と同じ
\end{thm}
\pause

\begin{proof}
証明も書ける。
\end{proof}
\pause

\begin{exam}                   %% 色が違う
example
\end{exam}
\end{frame}

\section{Relationship between $Q(s,a)$ function and $V(s)$}             %% 文字の色を変える
\begin{frame}
\frametitle{文字の色を変えてみよう}
{\color{red}赤}\pause
{\color{blue}青}\pause
{\color{green}緑}
\end{frame}

\end{document}

